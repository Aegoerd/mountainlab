\documentclass{article}
\usepackage[utf8]{inputenc}
\usepackage{geometry}
\geometry{margin=1in}

\usepackage{xcolor}
\usepackage{listings}
\lstset{basicstyle=\ttfamily,showstringspaces=false,commentstyle=\color{green},keywordstyle=\color{blue}}

\usepackage[hidelinks]{hyperref}

\title{MountainLab overview for SCDA - June 2016}
\author{Jeremy Magland}
\date{June 2016}

\begin{document}

\maketitle

\section{Introduction to MountainLab Components}

\begin{tabular}{|p{3cm}|p{5cm}|p{3cm}|p{4cm}|}
    \hline
    Component & Description & Where it runs & Dependency \\ \hline
    Matlab & Processing and Visualization within matlab environment & Workstation & Linux or OS X. Matlab, MountainSort, MountainView. \\ \hline
    Matlab wrappers & Matlab wrappers to MountainSort and MountainView. Can be mixed with the pure matlab component. & Workstation, Linux or OS X	& Matlab, MountainSort, MountainView \\ \hline
    MountainSort & Spike-sorting specific processing routines (Filter, whiten, cluster, etc.) & Workstation or Server, Linux or OS X & C++/Qt5, LAPACK \\ \hline
    MountainProcess & Processing engine (spike-sorting independent). Queue and run scripts and processes. Batch processing. & Usually Server, Linux & C++/Qt5 \\  \hline
    MountainView & GUI for visualizing raw data and spike sorting results. Either locally or remotely. & Workstation, Linux or OS X & C++/Qt5 \\ \hline
    MountainBrowser & Desktop GUI for browsing spike sorting result. This component will be replaced by a true website / HTML5 solution & Workstation, Linux or OS X & C++/Qt5 \\ \hline
    mpserver & Queue processing scripts on server & Server, Linux & nodejs \\ \hline
    mdaserver & Serve chunks of .mda files & Server, Linux & nodejs \\ \hline
    mbserver & Serve .json files & Server, Linux & nodejs \\ \hline
    mlproxy & Proxy server for mpserver, mdaserver, mbserver & Server, Linux & nodejs \\ \hline
\end{tabular}

\subsection{Conventions for spike sorting data}

The \textbf{raw dataset} (aka timeseries) is assumed to be a $M\times N$ array of voltage measurements, where $M$ is the number of channels (electrodes) and $N$ is the number of timepoints. Unless otherwise specified, event times are in index units.

The \textbf{sample rate} is the rate at which the original data was acquired. The samplerate variable is always in units of Hz. So samplerate=30000 means that timepoint 30,000 occurs at acquisition time 1 second.

Generally speaking, we use the following conventions:

\begin{itemize}
\item{\textbf{M} -- number of channels or electrodes}
\item{\textbf{N} -- number of timepoints in the timeseries}
\item{\textbf{K} -- number of clusters / spike types}
\item{\textbf{L} -- number of events / spikes}
\item{\textbf{T} -- length of a clip (in timepoints) / time window around a spike (typically 100 timepoints)}
\end{itemize}

One-based indexing is used for timepoints, channels, and cluster labels. Thus the first channel is channel $1$, the first timepoint is timepoint $1$, and the first cluster is $1$. A label of $0$ represents an unclassified spike event.

The \textbf{center of a clip} of size $T$ is always at the integer part of $(T+1)/2$ using one-based indexing.

The output of sorting is stored in a \textbf{"firings"} matrix of dimension $R\times L$ where $R\geq 3$ and $L$ is the number of events (double precision). Thus each column corresponds to a detected firing event. Information in the rows of the firings matrix are defined as follows:

\begin{itemize}
\item{\textbf{First row (optional)} -- Primary channel number of the spike type, e.g., the channel where the average spike shape has the largest peak. If this information is not provided (by the sorter) this row may be filled with zeros.}
\item{\textbf{Second row (mandatory)} -- Event times, or timepoints where the events were detected. These may have a fractional component.}
\item{\textbf{Third row (mandatory)} -- Integer labels assigning membership to a cluster or spike type. A value of zero indicates an unclassified (e.g., noise or outlier) event.}
\item{\textbf{Fourth row (optional)} -- Peak amplitudes, determined by the sorting algorithm.}
\item{\textbf{Fifth row (optional)} -- Outlier scores as determined by the sorting algorithm. The larger the score, the more outlier-ish.}
\item{\textbf{Sixth row (optional)} -- Detectability scores as determined by the sorting algorithm. Spike events that are close to noise are given a low score. Roughly speaking, a score of $6$ would correspond to an event that is $6$ standard deviations above the noise.}
\item{\textbf{Seventh row and up (optional))} -- Unused for now.}
\end{itemize}

\textbf{Electrode geometry} (geom.csv or geom.mda) is stored in a $2\times M$ or $3\times N$ array of 2D or 3D coordinates.

\subsection{Matlab and matlab wrappers}

There are four types of matlab functions for performing spike sorting and visualization.

\begin{itemize}
\item{\textbf{In-memory processing} - conventional Matlab processing functions}
\item{\textbf{In-memory visualization} - launches Matlab figures}
\item{\textbf{Wrappers to MountainProcess/MountainSort} - processing routines 
operating on files. Uses system calls to mountainprocess.}
\item{\textbf{Wrappers to MountainView} - visualization of raw data and results, operating on files. Uses (detached) system calls to mountainview.}
\end{itemize}

These four types can be used together in driver scripts via the following functions which convert between in-memory arrays and .mda files: arrayify.m, pathify8.m, pathify16.m, pathify32.m, pathify64.m. These are convenient wrappers to readmda.m and writemda*.m that return the inputs when a conversion is not needed. For example if X is an array, then arrayify(X) will simply return X.

\subsection{MountainSort}

MountainSort is a command-line program for performing low-level spike sorting routines. These functions operate on files in chunks, and therefore use very little RAM. Here are some examples:
\begin{lstlisting}[language=bash]
> mountainsort bandpass_filter --timeseries=raw.mda &&
--timeseries_out=filtered.mda --samplerate=30000 &&
--freq_min=300 --freq_max=2000

> mountainsort mask_out_artifacts --timeseries=filtered.mda &&
--timeseries_out=filtered2.mda --threshold=3 --interval_size=200

> mountainsort whiten --timeseries=filtered2.mda &&
--timeseries_out=whitened.mda

> mountainsort detect --timeseries=whitened.mda
--detect_out=detect.mda --clip_size=100 --detect_interval=10
--detect_threshold=3.5 --sign=-1 --individual_channels=1

> mountainsort branch_cluster_v2 --timeseries=whitened.mda
--adjacency_matrix=AM.mda --firings_out=firings.mda
--clip_size=100 --min_shell_size=150 --shell_increment=3
--num_features=10 --detect_interval=10

...
\end{lstlisting}

In general processes are defined based on the following data:
\begin{itemize}
\item{\textbf{processor\_name} -- The name or ID of the processor (e.g., bandpass\_filter)}
\item{\textbf{inputs} -- a collection of named input file paths}
\item{\textbf{outputs} -- a collection of named output file paths}
\item{\textbf{parameters} -- a collection of named input parameters (numbers or strings)}
\end{itemize}

\subsection{MountainProcess}

MountainProcess is the core, spike-sorting independent, component of MountainLab. It provides the following capabilities
\begin{itemize}
\item{\textbf{Plug-in processors} -- executables in any language}
\item{\textbf{Batch and parallel processing} -- intelligent queueing system for scripts and processes}
\item{\textbf{Provenance tracking and non-redundant process execution}}
\item{\textbf{Secure scripting using JavaScript}}
\end{itemize}

To take advantage of provenance tracking and queuing (batch) functionality all of the above mountainsort procedures may also be performed using mountainprocess from the command line. For example, bandpass filtering may be accomplished using
\begin{lstlisting}[language=bash]
> mountainprocess run-process bandpass_filter --timeseries=raw.mda &&
--timeseries_out=filtered.mda --samplerate=30000 &&
--freq_min=300 --freq_max=2000
\end{lstlisting}
or
\begin{lstlisting}[language=bash]
> mountainprocess queue-process bandpass_filter --timeseries=raw.mda &&
--timeseries_out=filtered.mda --samplerate=30000 &&
--freq_min=300 --freq_max=2000
\end{lstlisting}

The latter (queuing) command is useful for batch and parallel processing and requires the mountainprocess daemon to be running in the background. This daemon may be launched via
\begin{lstlisting}[language=bash]
> mountainprocess daemon-start
\end{lstlisting}
The recommend way to run the daemon in the background is using a tmux session:
\begin{lstlisting}[language=bash]
> tmux new -s mpdaemon
> mountainprocess daemon-start
\end{lstlisting}
Use [Ctrl b+d] to exit the tmux session, or simply close the terminal. Later you can attach via:
\begin{lstlisting}[language=bash]
> tmux attach -t mpdaemon
\end{lstlisting}
If you are also running the web servers, the start\_labcomputer.sh will also run the daemon (see below).

\subsection{MountainView}

MountainView is the desktop user interface for interactive visualization. Results may either reside on the local machine or on a remote server -- the functionality is the same. Detailed information on its usage can be found on the MountainLab forum in this article: \url{https://mountainlab.vbulletin.net/articles/23-introduction-to-mountainview}.

\subsection{Web servers: mpserver, mdaserver, mbserver}

To enable remote access to spike sorting results from client workstations, three servers must be running. 

The mountainprocess server (mpserver) enables

\section{Processing framework (MountainProcess)}

As mentioned above, MountainProcess is the core of MountainLab and provides a framework for processing data on a client workstation or server. It is completely independent of spike sorting. Plug-in processors may be created in any language as they are simply executables obeying a set of rules (specified below). It automatically handles batch and parallel processing using an intelligent system for queueing and running scripts and processes as resources become available. A crucial feature is provenance tracking and the ability to determine whether a particular process has already been performed. Processing may be performed either using a command-line/bash procedures or JavaScript-specified pipelines.

\subsection{Scripting processing pipelines}

The recommended alternative to writing bash scripts is to assemble the processing pipeline using JavaScript. MountainProcess can be used to queue or run these scripts. There are a number of advantages in terms of syntax, flexibility and security. First, convenience functions may easily be used to wrap commands. For example one way to perform pre-processing could be:
\begin{lstlisting}[language=C++]
X = bandpass_filter(X, opts);
X = mask_out_artifacts(X, opts);
X = whiten(X);
\end{lstlisting}

Another way, which is more intuitive in terms of intermediate files and multiple output parameters, is
\begin{lstlisting}[language=C++]
bandpass_filter(raw, '@pre1', opts);
mask_out_artifacts('@pre1', '@pre1b', opts);
whiten('@pre1b', '@pre2');
\end{lstlisting}
In this case the strings beginning with "@" are placeholders for intermediate files. The framework will automatically replace these strings with temporary file names that depend uniquely on the checksums of the input parameters. More details about unique file naming will be described later.

It is important to note that the structure of JavaScript function wrappers (including the above "@" intermediate file strategy) is completely independent of mountainprocess. There is a great deal of flexibility in how pipeline generation is performed.

The API for scripting is minimalistic and involves the following low-level interface methods:
\begin{itemize}
\item{\textbf{MP.fileChecksum(fname)} -- returns the SHA-1 hash of a local file (needed for purposes described below)}
\item{\textbf{MP.createTemporaryFileName(code)} -- returns the path of a temporary file, in a location managed by mountainprocess, which is uniquely determined by the input string.}
\item{\textbf{MP.runPipeline(json)} -- queues a pipeline of processes encoded by JSON text. This is simply a list of processes to execute, each with a processor name, input/output file names, and input parameters. The syntax is detailed below.}
\item{\textbf{MP.log(message)} -- writes a log message string to the console, for purposes of monitoring and debugging.}

\end{itemize}

\subsection{Plug-in processor libraries}

In MountainProcess, all data analysis is performed via plugin-processors library which are simply executables obeying a set of rules. First, the executable should have a .mp extension and should be placed in the mountainlab/mountainprocess/processors directory (although it is possible to configure mountainprocess to look in other directories as well). For spike sorting the processor library is mountainsort.mp. When this executable is run as
\begin{lstlisting}[language=bash]
> mountainsort spec
\end{lstlisting}
it should output JSON text that provides a list of available processors together with their specifications. For example, the following is a reduced version of the mountainsort.mp spec output:
\begin{lstlisting}[language=bash]
{
    "processors": [
        {"description": "","exe_command": "mountainsort.mp bandpass_filter $(arguments)",
            "inputs": [{"name": "timeseries"}],
            "name": "bandpass_filter",
            "outputs": [{"name": "timeseries_out"}],
            "parameters": [
                    {"name": "samplerate","optional": false},
                    {"name": "freq_min","optional": false},
                    {"name": "freq_max","optional": false},
                {"name": "processing_chunk_size","optional": true},
                {"name": "chunk_overlap_size","optional": true}
            ],
            "version": "0.1"
        },
        {"description": "","exe_command": "mountainsort.mp whiten $(arguments)",
            "inputs": [
                {"name": "timeseries"}
            ],
            "name": "whiten",
            "outputs": [{"name": "timeseries_out"}],
            "parameters": [],
            "version": "0.1"
        }
    ]
}
\end{lstlisting}

As outlined above, each processor has a name, and set of named input and output files, and a set of named input parameters (some optional). When running a process the arguments are passed in (by mountainprocess) according to the syntax from the section above (for mountainprocess). Note again the processing libraries are simply executable files and can therefore be created using any language.

\subsection{Multi-dimensional array file format (.mda)}

\subsection{File I/O and processing files in chunks}

\subsection{Executing processing scripts remotely}



\end{document}
